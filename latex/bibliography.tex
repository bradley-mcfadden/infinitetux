\begin{thebibliography}

\bibitem{blezinski2000}
Blezinski, C. (2000). The Art and Science of Level Design. Available at: https://www.gamedevs.org/uploads/the-art-science-of-level-design.doc [Accessed January 22, 2022]

\bibitem{browne2011}
Browne, C. (2011). Evolutionary Game Design.

\bibitem{charity2020}
Charity, M., Khalifa, A., & Togelius, J. (2020). Baba is y'all: Collaborative mixed-initiative level design. \emph{2020 IEEE Conference on Games (CoG)}, 542-549.

\bibitem{compton2006}
Compton, K., & Mateas, M. (2006). Procedural Level Design for Platform Games. 109-111.

\bibitem{dahlskog2012}
Dahlskog, S., & Togelius, J. (2012). Patterns and procedural content generation: Revisiting Marioo in world 1 level 1. \emph{Proceedings of the First Workshop on Design Patterns in Games}, 1-8. https://doi.org/10.1145/2427116.2427117

\bibitem{delarosa2021}
Delarosa, O., Dong, H., Ruan, M., Khalifa, A., & Togelius, J. (2021). Mixed-initiative level design with rl brush. \emph{International Conference on Computational Intelligence in Music, Sound, Art and Design (Part of EvoStar)}, 412-426.

\bibitem{guzdial2019}
Guzdial, M., Liao, N., Chen, J., Chen S.-Y., Shah, S., Shah, V., Reno, J., Smith, G., & Riedl, M. O. (2019). Friend, collaborator, student, manager: How design of an ai-driven game level editor affects creators. \emph{Proceedings of the 2019 CHI Conference on Human Factors in Computing Systems}, 1-13.

\bibitem{iyer1997}
Iyer, V., Bilmes, J., Wright, M., & Wessel, D. (1997). A novel representation for rhythmic structure. In \emph{Proceedings of the 23rd Internation Computer Music Conference}, 97-100.

\bibitem{jennings-teats2010}
Jennings-Teats, M., Smith, G., & Wardrip-Fruin, N. (2010). Polymorph: Dynamic difficulty adjustment through level generation. https://doi.org/10.1145/1814256.1814267

\bibitem{kazemi2009a}
Kazemi, D. (2009). Spelunky Generator Lessons. https://tinysubversions.com/spelunkyGen/

\bibitem{kazemi2009b}
Kazemi, D. (2009, September 29). Spelunky's Procedural Space. http://tinysubversions.com/2009/09/spelunkys-procedural-space/

\bibitem{lawson1997}
Lawson, B., & Loke, S. M. (1997). Computers, words and pictures. \emph{Design Studies, 18(2)}, 171-183. https://doi.org/10.1016/S0142-694X(97)85459-2

\bibitem{liapis2013}
Liapis, A., Yannakakis, G. N., & Togelius, J. (2013). Sentient sketchbook: computer assisted game level authoring.

\bibitem{lubart2005}
Lubart, T. (2005). How can computers be partners in the creative process: Classification and commentary on the Special Issue. \emph{International Journal of Human-Computer Studies, 63(4)}, 365-369. https://doi.org/10.1016/j.ijhcs.2005.04.002

\bibitem{lucas2019}
Lucas, S. M., & Volz, V. (2019). Tile Pattern KL-Divergence for Analysing and Evolving Game Levels. \emph{Proceedings of the Genetic and Evolutionary Computation Conference}, 170-178. https://doi.org/10.1145/3321707.3321781

\bibitem{mawhorter2010}
Mawhorter, P., & Mateas, M. (2010). Procedural level generation using occupancy-regulated extension. \emph{Proceedings of the 2010 IEEE Conference on Computatinal Intelligence and Games}, 351-358. https://doi.org/10.1109/ITW.2010.5593333

\bibitem{nielsen1994}
Nielsen, J. (1994). Usability Engineering. \emph{Morgan Kaufmann Publishers Inc}.

\bibitem{nielsen2005}
Nielsen, J. (2005). Ten usability heuristics. https://www.nngroup.com/articles/ten-usability-heuristics/

\bibitem{norman2013}
Normal, D. (2013). The design of everyday things: Revised and expanded edition. \emph{Basic books}.

\bibitem{ripamonti2016}
Ripamonti, L. A., Mannala, M., Gadia, D., & Maggiorini, D. (2016). Procedural content generation for platformers: Designing and testing FUN PLEdGE. \emph{Multimedia Tools and Applications, 76(4)}, 5001-5050. https://doi.org/10.1007/s11042-016-3636-3

\end{thebibliography}