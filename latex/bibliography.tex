\begin{thebibliography}{9}

% 1
\bibitem{blezinski2000}
Blezinski, C. (2000). The Art and Science of Level Design. Available at: https://www.gamedevs.org/uploads/the-art-science-of-level-design.doc (Accessed January 22, 2022)

% 2
\bibitem{browne2011}
Browne, C. (2011). Evolutionary Game Design.

% 3
\bibitem{charity2020}
Charity, M., Khalifa, A., \& Togelius, J. (2020). Baba is y'all: Collaborative mixed-initiative level design. \emph{2020 IEEE Conference on Games (CoG)}, 542-549.

% 4
\bibitem{compton2006}
Compton, K., \& Mateas, M. (2006). Procedural Level Design for Platform Games. 109-111.

% 5
\bibitem{dahlskog2012}
Dahlskog, S., \& Togelius, J. (2012). Patterns and procedural content generation: Revisiting Marioo in world 1 level 1. \emph{Proceedings of the First Workshop on Design Patterns in Games}, 1-8. https://doi.org/10.1145/2427116.2427117

% 6
\bibitem{delarosa2021}
Delarosa, O., Dong, H., Ruan, M., Khalifa, A., \& Togelius, J. (2021). Mixed-initiative level design with rl brush. \emph{International Conference on Computational Intelligence in Music, Sound, Art and Design (Part of EvoStar)}, 412-426.

% 7
\bibitem{guzdial2019}
Guzdial, M., Liao, N., Chen, J., Chen S.-Y., Shah, S., Shah, V., Reno, J., Smith, G., \& Riedl, M. O. (2019). Friend, collaborator, student, manager: How design of an ai-driven game level editor affects creators. \emph{Proceedings of the 2019 CHI Conference on Human Factors in Computing Systems}, 1-13.

% 8
\bibitem{iyer1997}
Iyer, V., Bilmes, J., Wright, M., \& Wessel, D. (1997). A novel representation for rhythmic structure. In \emph{Proceedings of the 23rd Internation Computer Music Conference}, 97-100.

% 9
\bibitem{jennings-teats2010}
Jennings-Teats, M., Smith, G., \& Wardrip-Fruin, N. (2010). Polymorph: Dynamic difficulty adjustment through level generation. https://doi.org/10.1145/1814256.1814267

% 10
\bibitem{kazemi2009a}
Kazemi, D. (2009). Spelunky Generator Lessons. https://tinysubversions.com/spelunkyGen/

% 11
\bibitem{kazemi2009b}
Kazemi, D. (2009, September 29). Spelunky's Procedural Space. http://tinysubversions.com/2009/09/spelunkys-procedural-space/

% 12
\bibitem{lawson1997}
Lawson, B., \& Loke, S. M. (1997). Computers, words and pictures. \emph{Design Studies, 18(2)}, 171-183. https://doi.org/10.1016/S0142-694X(97)85459-2

% 13
\bibitem{liapis2013}
Liapis, A., Yannakakis, G. N., \& Togelius, J. (2013). Sentient sketchbook: computer assisted game level authoring.

% 14
\bibitem{lubart2005}
Lubart, T. (2005). How can computers be partners in the creative process: Classification and commentary on the Special Issue. \emph{International Journal of Human-Computer Studies, 63(4)}, 365-369. https://doi.org/10.1016/j.ijhcs.2005.04.002

% 15
\bibitem{lucas2019}
Lucas, S. M., \& Volz, V. (2019). Tile Pattern KL-Divergence for Analysing and Evolving Game Levels. \emph{Proceedings of the Genetic and Evolutionary Computation Conference}, 170-178. https://doi.org/10.1145/3321707.3321781

% 16
\bibitem{mawhorter2010}
Mawhorter, P., \& Mateas, M. (2010). Procedural level generation using occupancy-regulated extension. \emph{Proceedings of the 2010 IEEE Conference on Computatinal Intelligence and Games}, 351-358. https://doi.org/10.1109/ITW.2010.5593333

% 17
\bibitem{nielsen1994}
Nielsen, J. (1994). Usability Engineering. \emph{Morgan Kaufmann Publishers Inc}.

% 18
\bibitem{nielsen2005}
Nielsen, J. (2005). Ten usability heuristics. https://www.nngroup.com/articles/ten-usability-heuristics/

% 19
\bibitem{norman2013}
Norman, D. (2013). The design of everyday things: Revised and expanded edition. \emph{Basic books}.

% 20
\bibitem{ripamonti2016}
Ripamonti, L. A., Mannala, M., Gadia, D., \& Maggiorini, D. (2016). Procedural content generation for platformers: Designing and testing FUN PLEdGE. \emph{Multimedia Tools and Applications, 76(4)}, 5001-5050. https://doi.org/10.1007/s11042-016-3636-3

% 21
\bibitem{rouse2004}
Rouse III, R. (2004). Game design: Theory and practice. \emph{Jones and Bartlett Publishers}.

% 22
\bibitem{schneiderman1983}
Schneiderman, B. (1983). Direct Manipulation: A Step Beyond Programming Languages. \emph{Computer, 16(8)}, 57-59.

% 23
\bibitem{shaker2011}
Shaker, N., Togelius, J., Yannakakis, G. N., Weber, B., Simizu, T., Hashiyama, T., Sorenson, N., Pasquier, P., Mawhorter, P., Takahashi, G., Smith, G., \& Baumgarten, R. (2011). The 2010 Mario AI Championship: Level Generation Track. \emph{IEEE Transactions on Computational Intelligence and AI in Games, 3(4)}, 332-347. https://doi.org/10.1109/TCIAIG.2011.2166267

% 24
\bibitem{shaker2012}
Shaker, N., Nicolau, M., Yannakakis, G. N., Togelius, J., \& O'Neill, M. (2011, September). Evolving levels for Super Mario Bros. using grammatical evolution. In \emph{2012 IEEE Conference on Computational Intelligence and Games (CIG)}(pp. 304-311). IEEE.

% 25
\bibitem{shaker2016}
Shaker, N., Togelius J., \& Nelson, M. J. (2016). Procedural Content Generation in Games: A Textbook and Overview of Current Research. \emph{Springer}. ISBN 978-3-319-42714-0.

% 26
\bibitem{smith2008}
Smith, G., Cha, M., \& Whitehead, J. (2008). A framework for analysis of 2D platformer levels. \emph{Proceedings of the 2008 ACM SIGGRAPH composium on Video Games}, 75-80. https://doi.org/10.1145/1401843.1401858

% 27
\bibitem{smith2010}
Smith, G., Whitehead, J., \& Mateas, M. (2010). Tanagra: A mixed-initiative level design tool. \emph{Proceedings of the Fifth International Conference on the Foundations of Digital Games}, 209-216. https://doi.org/10.1145/1822348.1822376

% 28
\bibitem{smith2011}
Smith, G., Whitehead, E., Mateas, M., Treanor, M., March, J., \& Cha, M. (2011). Launchpad: A Rhythm-Based Level Generator for 2-D Platformers. \emph{IEEE Transactions on Computational Intelligence and AI in Games}. https://doi.org/10.1109/TCIAIG.2010.2095855

% 29
\bibitem{smith2013}
Smith, G., Othenin-Girard, A., Whitehead, J., \& Wardrip-Fruin, N. (2013, November). Endless Web. In \emph{Ninth Artificial Intelligence and Digital Entertainment Conference}.

% 30
\bibitem{sorenson2010}
Sorenson, N., \& Pasquier, P. (2010). The Evolution of Fun: Automatic Level Design Through Challenge Modelling.

% 31
\bibitem{togelius2011}
Togelius, J., Yannakakis, G. N., Stanley K. O., \& Browne, C. (2011). Search-based procedural content generation: A taxonomy and survey. \emph{IEEE Transactions on Computational Intelligence and AI in Games, 3(3)}, 172-186.

\end{thebibliography}