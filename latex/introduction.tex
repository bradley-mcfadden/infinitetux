\section{Introduction}

Among all the tasks when designing a game, level design choices are some of the
most important \cite{blezinski2000, smith2008}. A well-designed level pushes a games' mechanics to its limits,
excites the player, and makes a good game great. A great example is the original
\emph{Super Mario Bros.} game. Only the first level sees the introduction of new
game mechanics, and afterwards new enemies are slowly introduced. However, the
number of units sold indicates that players found it to be a wortwhile and
rewarding game in sprite of the limited number of mechanics \cite{shaker2011}.

Level designers need to have knowledge across the spectrum of skills applicable
in game design \cite{blezinski2000}. They need an understanding of the game's mechanics, to know how
the artifical intelligence behaves, and to incorporate art assets into their 
levels in order to match the vision of the creative lead. Additionally, level
designers often have to switch between two considerations that constantly need
their attention: "Is this a good test of the skills that the player has
developed at this point in the game?" and "Does this obstacle fit with the
overall pacing and rhythm of the level?". Level designers frequentlly have to
use an iterative "modify and test" approach for certain game genres, as a small
change can sometimes have a large impact on the playability of a level \cite{smith2010}.

In spite of the importance of this task, the software for level designers has only
improved marginally over the years. In 2D games, the software is similar to many image
manipulation programs. For instance, in \emph{Super Mario Maker}'s level editor (which
can be seen in Figure 1), the level designer drags terrain around with the resize and
translate handles, or uses drag and drop to place entities onto the map from a palette.
A similar comparison can be drawn for 3D level editors such as \emph{Trenchbroom}. In
Trenchbrrom, the level designer builds geometry as if they were creating a 3D model, then
moves around entities as they please. Level editing in both cases is a direct manipulation
task, but far too often level editors lack feedback that could help the designers more
quickly identify and correct mistakes, to the detriment of the efficiency and pleasure of
associated with level editing as a direct manipulation task \cite{schneiderman1983}.