\chapter{Conclusions \& Future Work}

During this project, procedural generation was discussed in depth. Several approaches for
generating levels for 2D platformer games were examined. Additionally, the role of computers
in the creative process was discussed. Functionality and usability requirements were
researched to improve the utility of the designed application. From this initial research,
a system was implemented that would allow a level designer to work collaboratively to create
interesting InfiniteTux levels.

The goals of this project were to create a level editor that helps a game designer create
varied levels for a game quickly, by working collaboratively. In these regards, I consider
the system a success. The system allows the game designer to quickly iterate through
ideas by using a procedural generation apraoch that was shown to have increased variation
over the default generation algorithm for two of three tile sizes. By adding a few additions
to the chunk library, the user can augment the system to generate levels with much more
variety. The level editor provides the user to participate in the level generation process
with a fair amount of initiative being shared between the level designer and the system.
The user is able to modify the most basic unit that the generator works with - chunks -, and
is also free to modify the level on a per-tile basis. If the user is unhappy with the output,
they can immediately re-generate a level or a subsection of a level, to view more 
alternatives. A fair amount of power is still held by the designer. The designer must take
the initiative to start changes on a level, to tell the system when it should take its turn,
and make necessary adjustments. 

Most of the focus on the project was spent on level generation, when much more can go into
facilitating the designer's creativity than having varied levels to be inspired by, or
influence over the generation. One of the original visions of this project was to make
editing levels as easy as writing code in an integrated development environment (IDE). Many
IDEs provide the programmer with several different suggestions at a time. Warnings are
provided when the programmer may cause a runtime error, or performs an unsafe operation. 
Suggestions nudege the user toward btter style, or using newer and safer features.
Currently, the improved editor for InfiniteTux just provides warnins about potential
playability issues to the user. One could imagine that in the future, suggestions are given
about repositioning level elements to create a better rhythm, or using hills instead of
solid tiles to give the user multiple paths.

This work has contributed to the field of procedural generation, and of mixed-initiative
systems by incorporating a procedural generation system into a level editor to improve the
level editing task by reducing the creative and manual effort required by the user. The
implications of this contribution suggest that a similar approach could be applied to other
creative tasks such as drawing, the generation of music, or perhaps generating models of
terrain. A fair amount of effort has been devoted to the specific field of platformer games,
in order to gain an understanding of their structure and existing methods for procedurally
creating content for these games. If a collaborative system were to be developed for any of
the listed domains, the developers should also research existing generation methods in the
domain, in order to find an approach that would have similar strengths to that of the mORE
generator. A few notable strengths of mORE are its ability to overlap content, to be able to
work with the user by allowing its content library to be modified, and by taking turns with 
the user to generate and re-generate sections of a level.

Another implication of this project is that it would prove advantageous to be able to adapt
an existing content generator, instead of writing an implemenetation of a closed-source 
system. If this were done, much more time could be spent on the portion of the system that
collaborates with the user. This may allow more focus to be given to the collaboration 
between the user and the system, so that there could be a better dialogue between them. As
it stands, the current system is analytical, where instead it could provide interpretation
along with its analysis. However, the content generator part of the system would need to
have an understanding of the context of the content that it produces, which may hinder its
ability to quickly produce results and generate ideas for the user. Perhaps the generator
could be separated into two parts, where a content analyzer determines improvements that
could be made after the generator does its work. Either way, this contetn analyzer component
would be interesting to expand on in further research.

To expand on the system's ability to provide suggestions to the user, it would be
interesting if the user could first design several levels that the system analyses, and
when asked for suggestions, the system could use a metric like KL-Divergence to generate
levels until some threshold of acceptable difference from the user's levels is reached.
In other words, the system could analyse the designer's behaviour, and attempt to produce
levels or suggestions for areas that are nitentionally very different to what the designer
would produce themselves. This feature could serve to produce ideas for the designer that
they may not be able to think of themselves.

A common problem often encountered during the selection of the generator and during 
considerations of what extra information to provide the user with, is that is seems to be
very difficult to define what makes a level fun. A single level generation approach is not
dominant over others, as indicated in the wide variety of approaches explored during the
introduction of this report. More research could be done on both platformer games and level
generation approaches to determine both what is fun in a platformer game, and how these
findings can be applied to level generation to create engaging levels.

Future work could extend further than using user-defined chunks to generate levels.
Perhaps, the program could generate these chunks, or the user could act as a supervisor for
a system that learns to create interesting levels. The current system is highly context
dependent. For each game that it's used with, a new chunk library would need to be created
by a designer familiar with the game. However, the part of the algorithm that places level
geometry could require changes if some tiles or level components in another game are not
1x1 in size, as they are in InfiniteTux. Regardless, it would be interesting to see this
approach to level editing made more general so it could be applied to a variety of games, and
not just limited to 2D tile-based platformers.